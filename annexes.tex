\begin{appendices}
\chapter*{\textsc{Annexe A}}
	\addcontentsline{toc}{chapter}{\textsc{Annexe A}}		
	
	\begin{lstlisting}	
#include "types.c"
#include <string.h>
#include <stdlib.h>

void GenererBlocF (Transition * PremiereTransition,Place * PremierePlace, char LeFichierEnVHDL[MAX_NOM]) {

	Transition * trans = PremiereTransition;
   	Place * pla = PremierePlace;
	FILE * f1; //fichier du le bloc F
	FILE * f2; // fichier du package
	FILE * f3; // fichier de l'entite
	
	f1=fopen(LeFichierEnVHDL,"w");
	f2=fopen("package_MAL.vhd","w");
	f3=fopen("entity_place.vhd","w");
	
	//Remplissage du fichier package	
	fprintf(f2,"package MAL is \n \t component place_b \n \t \t port(ck, ra0_int, ra1_init, activer, desactiver : in std_logic ;\n \t \t \t marque : out std_logic) ;\n ");
	fprintf(f2,"\tend component ; \n");
	fprintf(f2,"end MAL ;\n\n");
	fclose(f2);
		
	//Remplissage du fichier ENTITY
	fprintf(f3,"entity place_b is\nport (ck \t\t: in std_logic ;\n");
	fprintf(f3,"\tra0_init, ra1_init \t: in std_logic ;\n");
	fprintf(f3,"\tactiver, desactiver \t: in std_logic ;\n");
	fprintf(f3,"\tmarque \t\t: out std_logic ;\n");
	fprintf(f3,"end place_b;\n\n");
	fprintf(f3,"architecture archi_place of place_b is\n");
	fprintf(f3,"begin\n");
	fprintf(f3,"\tprocess (ck, ra0_init, ra1_init, activer, desactiver)\n");
	fprintf(f3,"\tbegin\n");
	fprintf(f3,"\t\tif (ra0_init = '1') then marque <= '0';\n");
	fprintf(f3,"\t\telsif (ra1_init = '1') then marque <= '1';\n");
	fprintf(f3,"\t\telsif (ck'event and ck = '1') then \n");
	fprintf(f3,"\t\t\tif (activer = '1') then marque <= '1';\n");
	fprintf(f3,"\t\t\telsif (desactiver <= '1') then marque <= '0'; \n");
	fprintf(f3,"\t\t\tend if ;\n");
	fprintf(f3,"\t\tend if ;\n");
	fprintf(f3,"\tend process ;\n");
	fprintf(f3,"end archi_place ;\n\n");
	fclose(f3);

	//Remplissage du fichier Bloc F
	fprintf(f1,"------bloc F-------\n");

	int i,j,nb,o,same;	
	char *Nom_p;
	Arc *arc_So;
	Arc *arc_En;
	Arc *arc_N;

	//parcours de le liste Place
	do{  
		same=0;
	
		//Remplissage de la partie activation des places 
        	Nom_p=pla->Nom;	
		fprintf(f1,"a_%s <= '1' when ",Nom_p);	

		//parcours de le liste Transition
 		do{
			nb=trans->NbPlacesSortie;//retenir le nombre de places que va activer la transition actuelle 
    			o=trans->NbPlacesEntree; //retenir le nombre de places qui peuvent affranchir la transition actuelle 
    			arc_So=trans->ArcsSortants; //retenir le nombre d'arcs sortant de la transition actuelle 
    			arc_En=trans->ArcsEntrants; //retenir le nombre d'arcs entrant des places vers la transition actuelle 
				
			// parcours du champs "ArcsSortants" de la transition actuelle
        	  	for(i=0;i<nb;i++){ 
				if(strcmp (arc_So->Place ,Nom_p) == 0 ){
			      		if(o==0){
						if(same!=0) fprintf(f1," or "); // On reste sur la meme ligne 
        	                              	fprintf(f1,"(");same=1;
        	                              	fprintf(f1,"%s)",trans->Predicat);
        	             		}
				        else{ 
						if(same!=0) fprintf(f1," or "); // On reste sur la meme ligne 
        	                        	fprintf(f1,"(");
						
						// parcours de la liste Arc
        	                        	for(j=0;j<o;j++) {
        	                          		fprintf(f1,"%s='1'",arc_En->Place);
				            		fprintf(f1," and ");
			                    		arc_En=arc_En->Suivant; 
        	                            		same=1;
        	                       	 	}
				        	fprintf(f1,"%s)",trans->Predicat);
        	              		} 			
			   	}
        	    		arc_So=arc_So->Suivant; 	     	
        	 	} 
 			trans=trans->Suivant;
 		} while(trans!=NULL);

 		trans = PremiereTransition;
		fprintf(f1," else '0';");
		fprintf(f1,"\n"); 

		//Remplissage de la partie desactivation des places 		
		same=0; 
		Nom_p=pla->Nom;	
		fprintf(f1,"d_%s <= '1' when",Nom_p);
		
		//parcours de le liste Transition
		do{ 
 			o=trans->NbPlacesEntree;
 			arc_En=trans->ArcsEntrants;
		
			// parcours du champs "ArcsEntrants" de la transition actuelle
 			for(i=0;i<o;i++){
				if (strcmp (arc_En->Place ,Nom_p) == 0){
        	        		if(nb==0){ 	
			        		fprintf(f1,"%s",trans->Predicat);
						same=1;
        	                	}
			        	else {
        	               			if(same!=0)fprintf(f1," or "); // On reste sur la meme ligne
        	                     		fprintf(f1," (");
        	     	             		arc_N=trans->ArcsEntrants;
			
						// parcours de la liste Arc
        	                     		for(i=0;i<o;i++){
				      			fprintf(f1,"%s='1' and ",arc_N->Place);
        	                      			arc_N=arc_N->Suivant;
							same=1;
        	                        }
				     	fprintf(f1," %s) ",trans->Predicat);                 
					}  
        	       		}
        	       arc_En=arc_En->Suivant;			  
        	       }                                    
		trans=trans->Suivant;
		} while(trans!=NULL); 
	
		trans = PremiereTransition;
		fprintf(f1,"else '0';");
		fprintf(f1,"\n\n");
		pla=pla->Suivant; 	
		} while( pla !=NULL);

		pla = PremierePlace;		
		fprintf(f1,"--marquage:\n\n");
		
		// Remplissage de la partie marquage en parcourant la liste Place
		do{
			Nom_p=pla->Nom;
			fprintf(f1,"place_%s: place_b port map (ck, r_%s, s_%s, a_%s, d_%s, %s);\n",Nom_p,Nom_p,Nom_p,Nom_p,Nom_p,Nom_p);
			pla=pla->Suivant;
		} while(pla !=NULL);
		fclose(f1);	
}	
	\end{lstlisting}
\end{appendices}